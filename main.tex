\documentclass[a4paper, 10pt, oneside]{article}

% Essentials
\usepackage[utf8]{inputenc}
\usepackage[english]{babel}
\usepackage[hidelinks]{hyperref}
\usepackage[backend=bibtex,style=ieee]{biblatex}
\usepackage[tmargin=3cm, bmargin=2cm, lmargin=2cm, rmargin=2cm]{geometry}
\usepackage[backend=bibtex,style=ieee]{biblatex}

% Bibliography
%\bibliography{main}

% Optional but useful
\usepackage{amsmath,amsthm,amsfonts,amssymb}
\usepackage{graphicx}
\usepackage{multicol}
\usepackage{float}
\usepackage{subfiles}
\usepackage{enumitem}
\usepackage{tabularx,booktabs,multirow}
\usepackage{longtable}
\usepackage{url}
\usepackage{enumitem}
\usepackage{lipsum}
\usepackage{mdframed}
\usepackage{csquotes}
\usepackage[table]{xcolor}
\usepackage{fancyhdr}
\usepackage{titling}
\usepackage{listings}

% Fancy Header
\pagestyle{fancy}
\fancyhead[L]{\leftmark}
\fancyhead[R]{\thepage}
\fancyfoot{}

% Indentation
\setlength\parindent{0pt}

% Title
\title{Parallel Computing}
\date{\today}
\author{\textbf{Dieter Castel \& Jonas Devlieghere}}

\setlist[description]{style=nextline}
\setlength\parindent{0pt}
\graphicspath{{figures/}{../figures/}}

% Listings
\lstset{
	language=c,
	frame=lines,
	numbers=left,
	breakatwhitespace=false,
	captionpos=b
}

% Theorems
\theoremstyle{definition}
\newtheorem*{question}{Question}
\newtheorem*{solution}{Solution}
\begin{document}

% Title page
\maketitle
\newpage

% Table of Contents
\tableofcontents
\newpage

% Main Matter
\section{Exam 2014-01-24}

\subsection{One-to-All}
\begin{question}
Given are $P$ processors. Processor $p(0)$ sends an element to all processors. (one-to-all).
\begin{enumerate}
	\item Given the BSP program.
	\item What is the $h$-relation.
	\item How can this be improved, given that $l << g$.
\end{enumerate}
\end{question}
\begin{solution}
\end{solution}

\subsection{LU Factorization}
\begin{question}
LU factorization with 2-phase broadcast.
\begin{enumerate}
	\item Give the total sequential cost and its order.
	\item Give the total BASP cost and its order when the data is distributed over $\sqr{P} \times \sqr{P}$ processors.
	\item Given the order of the total cost when distributed over $1 \times P$ processors.
\end{enumerate}
\end{question}
\begin{solution}
\end{solution}

\subsection{Parallel Efficiency}
\begin{question}
Given a multicore program for calculating the average value of a matrix $A$. What is the efficiency and how can we do better?
\end{question}
\begin{solution}
\end{solution}

\subsection{BSP Quicksort}
\begin{question}
Give a high-level description of a BSP quicksort algorithm. For each step, calculate the cost ($a + bg + cl$). Calculate the total cost and give the overhead and efficiency. Discuss your results.
\end{question}
\begin{solution}
\end{solution}

\section{Exam 2014-01-17}

\subsection{Pairwise Swap}
\begin{question}
Consider p processes (p is even), each process has 1 data element. The even processes s swap their element with their neighbour s+1 (= pair-wise swap).
\begin{enumerate}
	\item Give a BSP algorithm.
	\item What is the $h$-relation?
	\item Suppose (non-BSP) 2-sided send and receive. When send transmits, hold execution until remote process receives. Then execute the communication and both processes continue. This is blocking pair-wise communication (as featured in MPI). Give the algorithm.
	\item Executing a send-receive communication has start-up time $s$, and data is transfered with bandwidth $b$. Is $s < l$? Will the cost be higher than the BSP algorithm from 1.1?
\end{enumerate}
\end{question}
\begin{solution}
\end{solution}

\subsection{SpMV}
\begin{question}
$A$ is a sparse $m$ by $n$ matrix, $x$ and $y$ are vectors.
\begin{enumerate}
	\item What is the sequential cost of $y=Ax$?
	\item Assume $A$ is distributed row-wise in a 1D fashion. Which phase, if any, of the classic SpMV phases (fan-out, SpMV, fan-in) will disappear? What will be the new total cost? What will be the parallel overhead?
	\item $H=(V,N)$ is a hypergraph of A. What hypergraph model would you use to distribute A using the previous 1D distribution type? Give definitions of $V$ and $N$ in that model.
	\item Assume SpMV algorithm 4.5 as in the book. Use your definition of $H$ from the previous question to write a cost function that measures the number of gets and puts.
\end{enumerate}
\end{question}
\begin{solution}
\end{solution}

\subsection{Odd-Even Transposition}
\begin{question}
Array $x$ has $n$ elements with $n$ a multiple of $p$. In odd-even transposition sort, the fundamental operation is compare_exchange (when each process has 1 element, $n=p$) or compare_split (each process has $n/p$ elements). Give for both cases:
\begin{enumerate}
	\item A BSP algorithm.
	\item Parallel execution time.
	\item Parallel efficiency.
	\item For n/p elements: if n increases (with a fixed p), what will happen to the parallel efficiency?
\end{enumerate}



\end{question}
\begin{solution}
\end{solution}

\subsection{Shared Memory}
\begin{question}
We have p threads running the same SPMD program on a shared memory computer. Will the following algorithm work? Describe any possible problems. Supply a working parallel algorithm.
\begin{lstlisting}[caption={SPMD program on a shared memory computer},label=lst:spmd]
double a;
double x[1000];
double y[1000];

void spmd(){
	for (i=0; i<1000; i+=p)
		a = a + x[i]y[i];
}
\end{lstlisting}

\end{question}
\begin{solution}
\end{solution}

% Appendix
\newpage
\appendix
\printbibliography

\end{document}
