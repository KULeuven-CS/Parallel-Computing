\documentclass[../main.tex]{subfiles}
\begin{document}

\subsection{One-to-All}
\begin{question}
Given are $P$ processors. Processor $p(0)$ sends an element to all processors. (one-to-all).
\begin{enumerate}
	\item Given the BSP program.
	\item What is the $h$-relation.
	\item How can this be improved, given that $l << g$.
\end{enumerate}
\end{question}
\begin{solution} The solution to the subquestions are given below.
\begin{enumerate}
	\item The BSP program for one-to-all communication is given below.
\begin{lstlisting}
void oneToAll(){
	bsp_begin(P);
	bsp_push_reg(element);
	for(pid=1; pid<P; pid++){
		bsp_put(pid, data, element, 0, sizeof(data));
	}
	bsp_sync();
	bsp_end();
}
\end{lstlisting}
	\item The $h$-relation is given by $h = max\{h_s,h_r\}$. In the previous case, $p(0)$ sends a single element to
	$(p-1)$ processors.
	\begin{equation}
		h = (p - 1)
	\end{equation}
	The BPS cost is given by $T_\text{comm}(h) = hg + l$, making the cost of this broadcast $(p-1)g + l$.
	\item If the cost of $l$ is significantly less than $g$, splitting the broadcast over multiple communication
	supersteps can reduce cost. Am example is a multi-level broadcast where every processor sends the element to two
	others. In $log_2(p)$ supersteps, every processor has received the element.
	\begin{equation}
		T_\text{comm} = \lceil log_2(P) \rceil (2g + l)
	\end{equation}
\end{enumerate}


\end{solution}

\subsection{LU Factorization}
\begin{question}
LU factorization with 2-phase broadcast.
\begin{enumerate}
	\item Give the total sequential cost and its order.
	\item Give the total BSP cost and its order when the data is distributed over $\sqrt{P} \times \sqrt{P}$
	processors.
	\item Given the order of the total cost when distributed over $1 \times P$ processors.
\end{enumerate}
\end{question}
\begin{solution} The solution to the subquestions are given below.
\begin{enumerate}
	\item The number of flops in the LU decomposition with row permutations is given in \autoref{eq:tseqlu}.
	\begin{equation}\label{eq:tseqlu}
		\begin{array}{ll}
		T_{seq} & = \displaystyle\sum_{k=0}^{n-1} (2(n - k - 1)^2 + n - k -1) \\
				& = \displaystyle\sum_{k=0}^{n-1} (2k^2 + k) \\
				& = \dfrac{(n-1)n(2n-1)}{3} + \dfrac{(n-1)n)}{2} \\
				& = \dfrac{2n^3}{3} - \dfrac{n^2}{2} - \dfrac{n}{6} \\
		\end{array}
	\end{equation}
	The sequential LU-decomposition algorithm is $\mathcal{O}(n^3)$.
	\item The number of flops in the LU decomposition with row permutations is given in \autoref{eq:tparlu}.
	\begin{equation}\label{eq:tparlu}
			T_{LU}  = \dfrac{2n^3}{3p} + \dfrac{3n^2}{2\sqrt{p}} + \dfrac{3n^2g}{\sqrt{p}} + 8nl
	\end{equation}
\end{enumerate}
\end{solution}

\subsection{Parallel Efficiency}
\begin{question}
Given a multicore program for calculating the average value of a matrix $A$. What is the efficiency and how can we do
better?
\end{question}
\begin{solution} The \emph{Parallel Efficiency} is defined in \autoref{eq:pareff}. The efficiency correlates the
increase in speedup with the number of processors. The efficiency of a given parallel algorithm can be increased by
either decreasing the number of processors (Amdahl's Law) or increasing the problem size.
\begin{equation}\label{eq:pareff}
	E = \frac{T_\text{sec}}{p \cdot T_p} = \frac{S}{p}
\end{equation}
\end{solution}

\subsection{BSP Quicksort}
\begin{question}
Give a high-level description of a BSP quicksort algorithm. For each step, calculate the BSPC cost ($a + bg + cl$).
Calculate the total cost and give the overhead and efficiency. Discuss your results.
\end{question}
\begin{solution}
\end{solution}

\end{document}